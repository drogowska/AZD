\documentclass[12pt]{article}
\usepackage[polish]{babel}
\usepackage{polski}
\usepackage[T1]{fontenc}
\usepackage[a4paper,top=2cm,bottom=2cm,left=3cm,right=3cm,marginparwidth=1.75cm]{geometry}
\usepackage{titling}
\renewcommand\maketitlehooka{\null\mbox{}\vfill}
\renewcommand\maketitlehookd{\vfill\null}
\newcommand{\BibTeX}{{\sc Bib}\TeX} 
\usepackage{amsmath}
\usepackage{graphicx}
\usepackage{float}
\usepackage{amsfonts}
\usepackage{url}
\usepackage{multirow}
\usepackage[colorlinks=true, allcolors=blue]{hyperref}
\usepackage{tabularray}
\usepackage{amssymb}
\usepackage[utf8]{inputenc}
\usepackage{longtable}
\usepackage[demo]{graphicx}
\usepackage{caption}

\title{Analiza złożonych danych z detekcją wyjątkówe   \linebreak
Zadanie nr. 1 – Klasyfikacja wzorców
  }
\author{Daria Rogowska, 249989  \\ 
Piotr Gwóźdź, 249948 }
\begin{document}
\begin{titlingpage}
\maketitle
\end{titlingpage}
\graphicspath{ {./pictures/} }
\newpage

\section*{Zbór danych}

Do przeprowadzenia eksperymentów wybrano zbiór danych [1] zwierający wyniki badania EEG. Dane zbierano od dwóch osób
 różnej płci przez 3 min na każdego ze stanów emocjonalnych - pozytwyny, neutralny i negatywny. Dodatkowo zarejestrowano dane
 zabrane z okresu 6 min neutralnego odpoczynku. W celu wywołania emocji o określonym charakterze badanym zaprezentowano filmy video. 

Baza danych zawiera 2132 rekordów opisanych w 2549 wartościach odczytnych z elektrod. 





\section*{Wnioski}


\section*{Literatura}
[1] Zbiór danych "EEG Brainwave Dataset: Feeling Emotions" \url{https://www.kaggle.com/datasets/birdy654/eeg-brainwave-dataset-feeling-emotions}



\end{document}